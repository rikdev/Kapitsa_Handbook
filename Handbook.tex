\documentclass[12pt,a4paper,titlepage]{article}

\usepackage[T1,T2A]{fontenc}
\usepackage[utf8]{inputenc}
\usepackage[russian]{babel}
\usepackage[unicode]{hyperref}


\begin{document}

\author{П.\,Л.~Капица}
\title{Карманный справочник физика - экспериментатора (цитатник)}
\date{1974}
\maketitle

\begin{abstract}
Предлагаемая читателю книга <<Карманный справочник физика - экспериментатора (цитатник)>> П.\,Л.~Капицы представляет собой своеобразный талмуд, охватывающий почти все разделы взаимодействия эксперимента с окружающей средой.
Большинство глав представляет собой сжатое изложение основных достижений Капицы и др. по затрагиваемым областям. книга заполняет пробел, существующий в литературе между Широко известным <<Кем быть>> В.\,В.~Маяковского и <<Афоризмами житейской мудрости>> Шопенгауэра.
Справочник рассчитан на студентов, аспирантов, научных и инженерно-технических работников различный специальностей. Много полезного найдут в нем лица, интересующиеся П.Л.Капицей.
\end{abstract}

\tableofcontents
\newpage

\section{Физика и жизнь}

\begin{enumerate}
\item <<Мы должны быть благодарны богу, что он создал мир так, что все простое правда, а все сложное - неправда>>.
\item Ничто так не поучительно, как заблуждение гения.
\item Национальная черта~--- у нас не любят эксперимент.
\item <<Богу и Маммоне служить одновременно нельзя>>.
\item Сколько у вас в общежитии ванных комнат?
\item Если человек сразу получает большую зарплату, то он не растет. Если же зарплата постепенно увеличивается, то человек достигает более высокого уровня. В русских сказках принцу приходится убивать дракона, преодолевать крепости и в награду он получает красивую принцессу. Если де он не хочет этого делать, то будет жить с простой бабой и избе.
\item Человек в своем развитии проходит три стадии. Первые 25 лет~--- это животное состояние. Человек думает главным образом о своих страстях и гораздо меньше о науке. Следующие 25 лет~--- смешанное состояние, ибо человек думает то об удовлетворении животных страстей, то о полезной деятельности и только следующие 25 лет можно считать человеческим состоянием. В человеке уже не бушуют страсти и он может посвятить себя полезной деятельности. Ну, а что касается тех 25, которые идут поле 75~--- то это божественное состояние. Человек становится иконой. Он ничего не делает, но на него молятся.
\item Из истории науки хорошо известно, что административным порядком правильно направлять развитие науки невозможно, даже в наши дни.
\item Физика во многих странах больна болезнью Паркинсона. Пока помещение и оснастка плохи, работа идет хорошо, но как только создаются хорошие условия, работа прекращается.
\item Среди ученых тоже есть болезнь, описанная в законе Паркинсона. Один из признаков~--- слишком много лаборантов.
\item Для науки нужны люди, которые прежде всего понимают, а для вуза~--- тот, кто больше всего знает.
\item По моему мнению, хороших инженеров мало. Они должны состоять из 4-х частей: на 25\% инженер должен бать теоретически образован, на 25\% он должен быть художником (машину нельзя проектировать, ее нужно рисовать~--- меня так учили и я тоже так считаю). На 25\% он должен быть экспериментатором, то есть исследовать свою машину и на 25\% он должен быть изобретателем. Вот так должен быть составлен инженер. Это очень грубо, могут быть вариации. Но все эти элементы должны быть.
\item <<Чем выше лезет обезьяна, тем выше ее зад>>.
\item <<Не имея ничего лучшего, спи с собственной женой>>.
\item Больному не нужно прекрасное лекарство, которое сделано уже после его смерти.
\item Нельзя думать, что создав в консерватории отделение по написанию гимнов и кантат, мы их получим. Если нет в этом отделении крупного композитора, равного по силе, например, Генделю, то все равно ничего не получится. Хромого не научишь бегать, сколько денег на это не трать.
\item <<Есть многое на свете, друг Горацио, что и не снилось нашим мудрецам>>.
\item Без бездельников не приживешь.
\item Конференции очень расплодились.
\item Если через чиновников действовать, то ничего не выйдет.
\item Конференции устраиваются для удовольствия людей и присутствие на них необязательно.
\item Академии наук существуют во всех странах. Интересно выяснить, зачем они нужны?
\item <<Я ее ужинаю, я ее и танцую>>.
\item Отчеты о командировках редко содержат сведения, за получением которых необходимо было и стоило ездить на место.
\item Роль руководителя исключительно велика. В современных условиях руководитель научной работы подобен режиссеру. Он создает спектакль, хотя и не появляется сам на сцене.
\item <<Никто не обнимет необъятного>>.
\item Только очень глупые люди не понимают шуток.
\end{enumerate}

\end{document}